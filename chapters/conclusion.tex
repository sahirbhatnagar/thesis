%-----------------------------------------------------------------------------
\chapter{Conclusion\label{ch:conclusion}}
%-----------------------------------------------------------------------------

\section{Summary}
The three manuscripts (Chapters~\ref{ch:sail},~\ref{ch:ggmix} and~\ref{ch:eclust}) presented in this thesis describe a body of work all within the context of penalized regression methods for interactions and mixed-effects models in high-dimensional data analysis. The first work (Chapter~\ref{ch:sail}) introduces a sparse additive interaction learning model called \sail ~for detecting non-linear interactions with a key environmental or exposure variable in high-dimensional settings. Through a simple re-parametrization, we demonstrate how our method can accommodate either the strong or weak heredity constraints. The second work (Chapter~\ref{ch:ggmix}) develops a general penalized linear mixed-model framework called \ggmix ~that simultaneously, in one step, selects variables and estimates their effects, while accounting for between individual correlations. We develop a blockwise coordinate descent algorithm which is highly scalable, computationally efficient and has theoretical guarantees of convergence. This work provides a roadmap for incorporating many other penalty functions in a mixed-model context. The final manuscript (Chapter~\ref{ch:eclust}) proposes a strategy for dimension reduction that leverages the effects of an exposure variable with
broad impact on high-dimensional measures. We show how these dimension-reduced variables, constructed without using the response, can be used in predictive models of any type. Furthermore, we demonstrate how environment-dependent correlations can induce an interaction model. 

The simulation studies in each chapter provide us with a better understanding of the strengths and limitations of each method developed in the thesis. In the first manuscript, the simulation study shows that \sail ~has very good performance in terms of both prediction error and yielding correct sparse models when the truth follows either strong or weak hierarchy as well when only main effects are present. The simulation study in the second manuscript demonstrates that in the context of variable selection in high-dimensional LMMs, existing approaches such as a two-stage method or the lasso with a principal component adjustment lead to a large number of false positives. We then show that \ggmix ~leads to correct Type 1 error control, improved variance component estimation, and is also robust to different kinship structures and heritability proportions. In the third manuscript, we found improved prediction and variable selection performance compared to methods that do not consider the environment in the clustering step, or to methods that use the original data as features.

The first manuscript contains an analysis of the Alzheimer's Disease Neuroimaging Initiative (ADNI) data using \sail, with the objective of selecting non-linear interactions between clinical diagnosis and A$\beta$ protein in the 96 brain regions on mini-mental state examination. This analysis selected both the middle occipital gyrus left region in the occipital lobe known for visual object perception, and the cuneus region which is known to be involved in basic visual processing. We found that more A$\beta$ protein loads in the middle occipital gyrus left region lead to a worse cognitive score for the MCI and AD group but not for the controls. For the cuneus region we found that more A$\beta$ proteins lead to better cognitive scores for the MCI and AD group and poorer scores for the controls. In the third manuscript, we applied \texttt{eclust} to three data sets, from very different fields, each with high dimensional data, a binary exposure, and a phenotype of interest. In the first data set, normal brain development was examined through brain imaging in conjunction with intelligence scores. In the second data set, we identified molecular subtypes of ovarian cancer using gene expression data. In the third data set, we examined the impact of gestational diabetes mellitus on DNA methylation and childhood obesity in a sample of mother-child pairs from a prospective birth cohort. The genes that were selected by \texttt{eclust} were implicated in several pathways related to the physiological system development and function.

A freely available and open source \texttt{R} software package has been created for all three methods developed in the thesis. We also provide extensive documentation to demonstrate a typical analysis pipeline for each package, and how all the functions are tied together. Each package has been tested on all three platforms (Mac, Windows, Linux), and a series of unit tests have been written for each package to ensure the quality of the code. We hope that our methods get included in future simulation studies that build upon this work. As statisticians, we believe our role is to provide analytic tools for applied researchers. In providing our code to the scientific community, we hope that the true potential of our work can be reached. 


\section{Future work}

The methods in Chapters~\ref{ch:sail} and~\ref{ch:ggmix} have only been developed for continuous responses. Further work on these topics would involve extending these models to binary outcomes, which are often measured in clinical settings. 

A limitation of \ggmix ~is that it first requires computing the covariance matrix with a computation time of $\mathcal{O}(n^2k)$ followed by a spectral decomposition of this matrix in $\mathcal{O}(n^3)$ time where $k$ is the number of SNP genotypes used to construct the covariance matrix. This computation becomes prohibitive for large cohorts such as the UK Biobank~\citep{allen2012uk} which have collected genetic information on half a million individuals. As a methodological improvement, it would be extremely useful to reduce this computational burden so that multivariable LMMs can be applied to the larger and more heterogeneous cohorts being assembled today.  

There are also methodological advancements that can be made to the \texttt{eclust} approach. For example, extending \texttt{eclust} to handle continuous exposures or multiple exposures. The best way to construct an exposure-sensitive distance matrix that can be used for clustering is not obvious in these situations. 

\section{Concluding remarks}

Accurate prediction and understanding which variables improve that prediction are two very challenging and overlapping problems in analysis of high-dimensional data, such as those arising in medical imaging, genomics, and neuroscience. Penalized regression methods are a beneficial modeling choice for high-dimensional data for a number of reasons, most notably due to the fact that we must often assume a sparse model because there is not enough information to estimate so many parameters. Given the vast amounts of data being generated at a rate faster than we can analyze it, there is a great need for faster and more efficient algorithms with good convergence properties. High-quality software implementations of these algorithms are likely to promote widespread usage of penalized regression methods by researchers in the applied sciences. 