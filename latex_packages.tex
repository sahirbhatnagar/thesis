%\documentclass[12pt,Bold,letterpaper]{mcgilletdclass}
\usepackage{amsfonts}
\usepackage{amsmath,amsthm,amssymb,bbm,mathrsfs,mathtools,xfrac} %math stuff
\usepackage{comment}
\usepackage{apacite} % this has to go before natbib package
%\usepackage[numbers,sort]{natbib}   % omit 'round' option if you prefer square brackets
\usepackage[sort]{natbib}   % omit 'round' option if you prefer square brackets



%\usepackage{algorithm}


%\usepackage{algorithmic}
%\usepackage[ruled,vlined,noresetcount]{algorithm2e}

% THIS (algorithm AND algpseudocode) ARE NEEDED FOR SAIL MANUSCRIPT TO WORK
\usepackage{algorithm,algorithmicx}
\usepackage[noend]{algpseudocode}
\algrenewcommand\textproc{}% Used to be \textsc
\algdef{SE}[SUBALG]{Indent}{EndIndent}{}{\algorithmicend\ }%
\algtext*{Indent}
\algtext*{EndIndent}
%%%%%%%%%%%%%%%%%%%%%%%%%%%%%%%%%%%%%


% THIS (algorithm2e) IS NEEDED FOR ggmix
%\usepackage[algo2e,ruled,vlined,noresetcount]{algorithm2e}
%%%%%%%%%%%%%%%%%%%%%%%%%%%%%%%%%%%%%%%%%%


\usepackage[toc]{appendix}
%\usepackage{appendix}
\usepackage{caption}
\usepackage{tcolorbox} % for box around text
%\usepackage{color}
\usepackage{color,colortbl,xcolor}
\usepackage{float} % for H in figures and tables
%\usepackage{enumitem}
%\usepackage{epsfig}
\usepackage{epstopdf}
\usepackage{framed}
\usepackage[letterpaper, margin=1in]{geometry}		% Margins should be 1 inch according to McGill requirements (https://www.mcgill.ca/gps/thesis/guidelines/preparation)
\usepackage{graphicx}
\usepackage{listings}
\usepackage{placeins}
\usepackage{booktabs}
\usepackage{threeparttable}
\usepackage{tabulary}



% used for tables
\def\@tab@fn#1{\ensuremath{^{\mbox{{\scriptsize #1}}}}}
\def\tabfnm#1{\rlap{\@tab@fn{#1}}}
\def\tabfnt#1#2{\raggedright\@tab@fn{#1}#2}

\def\@tab@fn#1{\ensuremath{^{\mbox{{\scriptsize #1}}}}}
\def\tabfnm#1{\rlap{\@tab@fn{#1}}}
\def\tabfnt#1#2{\raggedright\@tab@fn{#1}#2}


\let\tnote\relax% "Free up" \tnote for use in ctable
%\usepackage{setspace}
%\usepackage{subcaption}
\usepackage[utf8]{inputenc}
\usepackage[T1]{fontenc}
\usepackage{pifont}% http://ctan.org/pkg/pifont
\newcommand{\cmark}{\ding{51}}%
\newcommand{\xmark}{\ding{55}}%
\def\widebar#1{\overline{#1}}

\usepackage{siunitx}
\sisetup{output-exponent-marker=\ensuremath{\mathrm{e}}}

\usepackage{array}
%\newcolumntype{L}{>{\centering\arraybackslash}m{3cm}} % used for text wrapping in ctable
\usepackage[pagebackref=true,bookmarks]{hyperref}
\hypersetup{
	unicode=false,
	pdftoolbar=true,
	pdfmenubar=true,
	pdffitwindow=false,     % window fit to page when opened
	pdfstartview={FitH},    % fits the width of the page to the window
	pdftitle={Penalized LMM in Families},    % title
	pdfauthor={Sahir Rai Bhatnagar},     % author
	pdfsubject={Subject},   % subject of the document
	pdfcreator={Sahir Rai Bhatnagar},   % creator of the document
	pdfproducer={Sahir Rai Bhatnagar}, % producer of the document
	pdfkeywords={}, % list of keywords
	pdfnewwindow=true,      % links in new window
	colorlinks=true,       % false: boxed links; true: colored links
	linkcolor=red,          % color of internal links (change box color with linkbordercolor)
	citecolor=blue,        % color of links to bibliography
	filecolor=black,      % color of file links
	urlcolor=blue           % color of external links
}

%########################################################################################
%            						SPACING
%########################################################################################

\usepackage[parfill]{parskip} % Activate to begin paragraphs with an empty line rather than an indent
%\usepackage[left=.1in,right=.1in,top=.1in,bottom=.1in]{geometry}
%\usepackage[margin=1in]{geometry}
\usepackage{setspace}
\doublespacing

%\usepackage[figurename=Fig.]{caption}
\usepackage{subfig}
\usepackage{tikz, pgfplots}
\usetikzlibrary{arrows,shapes.geometric}
\usetikzlibrary{calc}
\usetikzlibrary{backgrounds,intersections,fit}
\tikzset{isometricYXZ/.style={x={(1cm,0cm)}, y={(-1.299cm,-0.75cm)}, z={(0cm,1cm)}}}
\newcommand*\ab{.4}

\usepackage{ctable} % load after tikz. used for tables



\definecolor{whitesmoke}{RGB}{245, 245, 245}
